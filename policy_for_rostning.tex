\documentclass{dsekprotokoll}

\usepackage[T1]{fontenc}
\usepackage[utf8]{inputenc}
\usepackage[swedish]{babel}
\usepackage{multicol}


\setheader{Policy för Röstning}{Policydokument}{Lund -- 27 november 2018}

\title{Policy för röstning}
\author{Noah Mayerhofer}

\begin{document}

\maketitle
\section{Formalia}
\subsection{Sammanfattning}
Policyn beskriver hur röstning går till på sektionen, och de olika formerna vilket detta kan göras genom.
\subsection{Syfte}
Det har tidvis förekommit diskussioner kring hur beslutsprocessen ska se ut på sektionsmötet,
främst gällande hur blank, ogiltiga och nedlagda röster definieras och räknas.


Många detaljer kring detta är definierat i stadgarna för TLTH. För att göra det enklare att följa
dessa bestämmelser sammanfattas de relevanta delarna i denna policy.


Vidare fastställs det i kårens stadgar att enkel majoritet, det vill säga pluralitet, räcker för beslut
bortsett från val. Denna policy fastställer att majoritet gäller istället.

\subsection{Omfattning}
Sektionen omfattas i sin helhet.

\subsection{Historik}
Policyn är antagen på HTM2 2018.

Uppdaterad enl. Policy för Policyer på HTM2 2021 av Kaspian Jakobsson.

Uppdaterad enl. Policy för styrdokument på VTM-extra 2023.


\section{Sammanfattning av TLTH:s stadgar}
Följande är en sammanfattning av de bestämmelser i TLTH:s stadga som berör beslutsproceduren på sektionsmötet. Informationen är ett urval, och baserat på tolkningar av texten i stadgan.

\subsection{Begrepp och förtydliganden}

Följande begrepp är definierade i TLTH:s stadgar.

\begin{itemize}
	\item \textbf{ Enkel majoritet – pluralitet,} det vill säga det alternativ med flest röster. För att undvika
	      ihopblandning med begreppet majoritet används pluralitet i det här dokumentet.
	\item \textbf{Kvalificerad majoritet} – minst två tredjedelar av de avgivna rösterna.

\end{itemize}


Begreppet majoritet definieras inte uttryckligen, men hälften av de avgivna rösternaänvänds.
För att vara konsekvent kommer begreppet majoritet att syfta på minst hälften av de avgivna
rösterna i det här dokumentet
\subsection{Votering}
Vid votering antecknas röstsiffrorna i protokollet. I personval antecknas de enbart vid begäran.

\subsubsection{Sluten Votering}

Sluten votering sker genom upprop enligt röstlängden och med röstsedlar.

Sluten votering kan också ske digitalt, antingen via inbyggda pollfunktioner i program för videokonferenser eller ett formulär på exempelvis Google Forms.

\subsubsection{Öppen Votering}

Öppen votering sker med handuppräckning. Vid begäran skall röstprotokoll upprättas. Det
sker då genom upprop enligt röstlängden. Röstprotokollet antecknas i protokollet.

Öppen votering kan också ske digitalt via ja/nej reaktioner i program för videokonferenser

\subsection{Val}
Vid personval skall votering vara sluten.


Om det finns färre kandidater än platser sker endast en omröstning, och pluralitet räcker.


Om det finns två kandidater till en post räcker pluralitet. Vid lika avgör lotten.


Vid personval med flera kandidater väljs den person som fått flest röster, men majoritet krävs.
Om antalet kandidater som fått majoritet av rösterna är mindre än antalet som skall väljas,
genomförs en andra omgång röstning.


I denna andra omgången deltar de kandidater som fått högst antal röster utan att få majoritet.
Antalet som kan delta i den andra omgången är dubbla antalet av återstående platser. Pluralitet
gäller i andra omgången.

\subsubsection{Rösträkning}
En röstsedel med fler namn än antalet som skall väljas räknas som ogiltig.


Namnet på en röstsedel räknas som obefintligt om namnet avser en person som ej är nominerad
eller om namnet inte tydligt anger vem som menas

\subsection{Övriga beslut}
Vid beslut andra än personval skall votering vara öppen.

Om inget annat stadgas krävs pluralitet.

Vid lika har talmannen utslagsrösten.

Ogiltiga röstsedlar ses som ej avgivna.

I beslut andra än personval krävs majoritet om inget annat stadgas.

\end{document}