\documentclass[]{dsekprotokoll}

\usepackage[T1]{fontenc}
\usepackage[utf8]{inputenc}
\usepackage[swedish]{babel}
\usepackage{float}
\usepackage{tabularx}
\usepackage{array}

\setheader{Policy för Mötestider}{Policydokument}{Lund -- 2021}

\title{Policy för Mötestider}
\author{Sean Jentz}

\begin{document}

\section{Formalia}

\subsection{Sammanfattning}
Denna policy beskriver de avgränsningar för mötestider som bör beaktas vid kallelser till sektionsmöten eller styrelsemöten.

\subsection{Syfte}
Syftet med denna policy är att reglera vilka mötestider som anses lämpliga för möten vilka är öppna för alla sektionsmedlemmar, specifikt sektionsmöten och styrelsemöten.

\subsection{Omfattning}
Riktlinjerna är gällande för D-sektionens mötestider med avseende på sektionmöte samt styrelsemöte.

\subsection{Ägande}
Sektionsmötet äger policyn i sin helhet och har rätt att behandla och ändra denna på sektionsmöten.

\subsection{Historik}
Ursprungligen antagen enligt beslut: HTM1 2016. \\
Omarbetning fastställd enligt beslut (samt motionär): \\
HTM1 2021, Sean Jentz, Aktivitetsansvarig 2021. \\
HTM1 2023, Framtidskutskottet.

\section{Riktlinjer för Mötestider}
För att alla sektionens medlemmar skall ha en möjlighet att påverka sektionens beslut samt få möjligheten att yttra angående sektionens verksamhet, så bör tiderna för sektionens öppna möten begränsas för att se till att alla medlemmar i största grad kan medverka.

\subsection{Sektionsmöte}
Mötestiderna för sektionsmöten bör begränsas till att hålla på mellan tiderna 17:00 -
23:00, senare än kl 23:00 bör pågående punkt avslutas med max en timmes tillägg.
Notera att mötestider för sektionsmötet även regleras i stadgarna, under kapitel 8. Om sektionsmötet avslutas med kvarvarande punkter så ajourneras mötet till nästkommande vardag, alternativt utsatt ajourneringsdag specificerat i kallelsen.

Om sektionsmötet beslutar att fortsätta mötet bör frågan om att ajournera mötet lyftas
efter varje avslutad punkt.

\subsection{Styrelsemöte}
Mötestiderna för styrelsemöten bör begränsas till ordinarie läsperiod, exkluderat tentaperioder. Tiderna 08.00 till 12.00 samt 13.00 till 17.00 på vardagar bör även dem tas hänsyn till då dessa ofta innefattar obligatoriska moment.

\subsubsection{Undantag för Styrelsemöte}
\begin{itemize}
    \item Styrelsemöten kan vid behov tillkallas veckan innan eller efter tentaperioder.
    \item Styrelsemöten som inte är öppna för sektionsmedlemmar, såsom möten bakom lykta dörrar, kan undantas från policyn.
\end{itemize}

\end{document}