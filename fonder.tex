\documentclass{dsekprotokoll}
\usepackage[T1]{fontenc}
\usepackage[utf8]{inputenc}
\usepackage[swedish]{babel}
\usepackage{multicol}

\setheader{Policy för fonder}{Policydokument}{}

\title{Policy för fonder}
\author{Fred Nordell}

\begin{document}

\section*{Policy för Fonder}
\section{Formalia}
\subsection{Sammanfattning}
Policyn beskriver hur och varför man avsätter pengar till större projekt i fonder.
\subsection{Syfte}
Fonder avser i denna policy konton med medel avsatta till ett särskilt ändamål och ej sparande i aktier. Sektionen har behov av att avsätta pengar till större projekt samt hantera uppkomna kostnader som inte tagits hänsyn till i budgeten. Den här policyn är till för att vägleda sektionens sparande i fonder.
\subsection{Omfattning}
D-sektionen som helhet
\subsection{Ägande}
Sektionsmötet äger policyn
\subsection{Historik}
Policyn är antagen HTM2 2018.
Uppdaterad HTM1 2020.
Uppdaterad enl. Policy för policyer HTM2 2021.
Uppdaterad HTM1 2022.

\section{D-sektionen fonder}
Sektionen har ett antal fonder för att kunna täcka oförutsedda utgifter eller de utgifter som är svåra att budgetera för. 
För samtliga fonder finns det beskrivet huruvida fondens medel disponeras av styrelsebeslut eller beslut från sektionsmötet
samt hur mycket pengar fonden maximalt skall innehålla. Även syftet med fonden skall framgå av fondens beskrivning. 
Slutligen skall även den summa som bör avsättas årligen finnas med i beskrivningen av fonden. Sektionens årliga budget 
bör avsätta medel till de fonder som behöver enligt de riktlinjer som finns i fondens beskrivning.
Fonderna skall sedan fyllas på med resultatet från det föregående verksamhetsåret i samband med bokslutet. 
I händelse av att en fond redan är fylld då pengarna skall överföras bör de budgeterade pengarna som överskrider maxtaket 
för fonden läggas till resultatet. Samtliga uttag ur någon av fonderna skall redovisas för nästkommande sektionsmöte. 
D-sektionens fonder skall fyllas på i hierarkisk ordning i den ordning som de är uppräknade under paragraf §2.1.

\subsection{Sektionsfond}
Syftet med sektionsfonden är att sektionen ska kunna göra långsiktiga investeringar i inventarier och lokaler. Fonden disponeras av styrelsen. Uttag ur fonden ska presenteras
till sektionsmöte.

Fonden består av de medel som avsattes av beslut på HTM-1 2018; Ytterligare medel tillskjutna av
sektionsmöte eller annan genom frivilliga bidrag.

Fonden disponeras av Sektionsstyrelsen. Uttag ur fonden ska presenteras till sektionsmöte.

\section{Fondavsättningstrategi}
D-sektionen fondavsättningsstrategi bör användas som underlag för verksamhetens resultatsdispositioner, beroende på hur D-sektionen ekonomiska resultat ser ut.


\paragraph{Scenario 1}  När D-sektionen genererar mer överskott än fonduttagen gjorda för verksamhetsåret ska D-sektionen i första hand disponera överskottet till de fonderna där medel använts.
Disponeringen ska motsvara summan av de använda medlen. Resterande överskott ska disponeras av
sektionsmötet.


\paragraph{Scenario 2} När D-sektionen genererar överskott som är lägre eller motsvarande fonduttagen
gjorda för verksamhetsåret ska sektionsmötet disponera överskottet till de fonder som anses lämpliga.

\section{Etiska hänsynstagningar och risker}

D-sektionen bör följa Teknologkårens policy för ekonomi när det kommer till etiska hänsynstagningar och riskaspekter kring sparande.




\end{document}