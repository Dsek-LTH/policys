\documentclass{dsekprotokoll}

\usepackage[T1]{fontenc}
\usepackage[utf8]{inputenc}
\usepackage[swedish]{babel}
\usepackage{multicol}

\setheader{Policy för tackverksamhet}{Policydokument}{}

\title{Policy för tackverksamhet}

\author{Anna Qvil, Fred Nordell}

\begin{document}

\maketitle
\section{Formalia}
\subsection{Sammanfattning}
Policyn beskriver hur tackverksamhet får gå till på sektionen och vilka funktionärsposter som har rätt till specifika fördelar.

\subsection{Syfte}
Sektionens arbete bygger på ideellt arbete av våra medlemmar och funktionärer. På grund av detta är det viktigt att våra funktionärer känner sig uppskattade. Tack har varit en punkt med delade åsikter under åren och denna policy ämnar att konkretisera D-sektionens syn på tackverksamhet.

\subsection{Omfattning}
Sektionen i sin helhet.

\subsection{Historik}
Policyn är antagen på HTM 2 2016 och uppdaterad HTM2 2018, HTM2 2019, VTM 2020, HTM1 2020.
Uppdaterad enl. Policy för Policyer på HTM2 2021 av Kaspian Jakobsson.
Uppdaterad HTM1 2022 och VTM 2023.
Uppdaterad enl. Policy för styrdokument på VTM-extra 2023.

\subsection{Definitioner}
I detta dokument defineras \textit{tackgrupp} som en gruppering
av funktionärer inom sektionen som har en egen tackbudget.



\section{Syfte}
Syftet med denna policy är att ge direktiv för hur funktionärer bör  bli tackade för arbetet de utfört i sektionens namn. Denna policy behandlar också hur mycket var tackgrupp ska få i tackbudget samt att vägleda i vad huvudsyftet för en tackverksamhet bör vara.
\section{Tackverksamhet}
Tackverksamheten på D-sektionen skall ej ha alkohol som huvudsyfte. Den skall fokusera på att ge ett tack som alla kan delta i oavsett bakgrund. Varje funktionärs jobb är lika värt och tackgrupperna får därmed lika mycket i budget för tackverksamhet. Vidare budgeteras det för arbetskläder för poster som anses ha mer ansvar över sektionens verksamhet då dessa bör utmärkas för deras extra insats, i likhet med medaljer etc.
\subsection{Förmåner}
\par Nedan regleras vad var sektionsmedlem äga rätt till samt vad de tillför till budgetering.
\par Var enskild funktionär, inte funktionärspost, bidrar med 400kr till tackgruppens tackbudget det året. Varje styrelseledamot som även är utskottsordförande bidrar till tackbudgeten för sitt respektive utskott och inte styrelsen. Dock kan en enskild person inte tackas mer än vad lagen och kårens policys reglerar, om detta inträffar skall tacket för personen i fråga diskuteras med denne och en lösning finnas där den känner sig tillfredsställd med det tack denne får.
\par Phaddrar, Funktionärer inom Sexmästeriet och Projektgruppen för Teknikfokus särbehandlas. Tacket för dessa funktionärer budgeteras på en separat budgetpost som inte är beroende av antalet funktionärer. Dessa tackkostnader får dock ej överstiga 400 kr per funktionär. Dessutom får de även rätt till nedanstående förmåner.

Var funktionär på D-sektionen äga rätt till:
\begin{attlista}
  \item delta på sektionsgemensamma tackevent.
  \item gå gratis på skiphtesgasquen.
  \item nyttja gratis kaffe och te i iDét.
\end{attlista}

\subsection{Arbetskläder}
\par Arbetskläder som används i D-sektionens namn skall följa de lagar som finns angående utformning och syfte. Vidare skall kläder som används med D-sektionens varumärke bäras med detta i åtanke, funktionärer skall vara medvetna om att de representerar sektionen när de använder dessa plagg.
Subventionering av arbetskläder motsvarande 350 kr ges till funktionärer som har en större ansvarspost.

\par Är en person funktionär inom två olika tackgrupper får denne subventionering för båda posterna. Om personen innehar två poster inom en tackgrupp får denne enbart subventionering för ett klädesplagg. Styrelsen får subventionerade styrelsetröjor utöver mästerispecifika tröjor för att det är viktigt att visa den dubbla tillhörighet som de innehar.

\subsection{Funktionärsposter med rätt till subventionering av tröja}
\begin{multicols}{2}
  \subsubsection*{Sexmästeriet}
  \begin{itemize}
    \item Sexmästare
    \item Vice Sexmästare
    \item Hovmästare
    \item Pubmästare
    \item Vice Pubmästare
    \item Barmästare
    \item Vice Barmästare
    \item Köksmästare
    \item Vice Köksmästare
    \item Preferensmästare
    \item Sångförman
  \end{itemize}
  \subsubsection*{Cafémästeriet}
  \begin{itemize}
    \item Cafémästare
    \item Vice cafémästare
    \item Dagsansvarig
  \end{itemize}
  \subsubsection*{Skattmästeriet}
  \begin{itemize}
    \item Skattmästare
    \item Vice skattmästare
    \item Skattförman
  \end{itemize}
  \subsubsection*{Informationsutskottet}
  \begin{itemize}
    \item Informationsansvarig
    \item Vice Informationsansvarig
    \item Dwww-ansvarig
  \end{itemize}
  \subsubsection*{Nollningsutskottet}
  \begin{itemize}
    \item Øverphøs
    \item Stabsmedlem
    \item \O verpeppare
    \item Peppare
  \end{itemize}
  \subsubsection*{Studierådet}
  \begin{itemize}
    \item Studierådsordförande
    \item Vice studierådsordförande
    \item Studierådssekreterare
    \item Årskursrepresentanter
  \end{itemize}
  \subsubsection*{Källarmästeriet}
  \begin{itemize}
    \item Källarmästare
    \item Vice Källarmästare
    \item Root
    \item Vice root
  \end{itemize}
  \subsubsection*{Trivselrådet}
  \begin{itemize}
    \item Trivselmästare
    \item Likabehandlingsombud
    \item Skyddsombud
    \item Världsmästare
          %Kanske bör ha med fler i mästeriet, de gör inte så mycket i tid men viktigt att veta vilka de är samt har de mycket ansvar%
  \end{itemize}
  \subsubsection*{Aktivitetsutskottet}
  \begin{itemize}
    \item Aktivitetsansvarig
    \item Vice Aktivitetsansvarig
    \item Utedischoansvarig
    \item Karnevalsansvarig
    \item Sångarstridsförman
  \end{itemize}
  \subsubsection*{Näringslivsutskottet}
  \begin{itemize}
    \item Näringslivsansvarig
    \item Vice Näringslivsansvarig
    \item Alumniansvarig
    \item Mentorsansvarig
  \end{itemize}
  \subsubsection*{Framtidsutskottet}
  \begin{itemize}
    \item Framtidsordförande
    \item Framtidsledamot
  \end{itemize}

  \subsubsection*{Valberedningen}
  \begin{itemize}
    \item Valberedningsordförande
    \item Valberedningsrepresentant
  \end{itemize}
  \subsubsection*{Övriga}
  \begin{itemize}
    \item Revisor
    \item Talman
  \end{itemize}

\end{multicols}


\end{document}