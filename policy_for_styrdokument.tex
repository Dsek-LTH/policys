\documentclass[]{dsekprotokoll}

\usepackage[T1]{fontenc}
\usepackage[utf8]{inputenc}
\usepackage[swedish]{babel}
\usepackage{titlesec}

\setheader{Policy}{Lund, \today}{}

\title{Policy för Styrdokument}
\author{Oliver Levay}

\begin{document}
\section*{Policy för styrdokument}
\section{Formalia}
\subsection{Sammanfattning}
Denna policy beskriver hur styrdokument på D-sektionen fungerar samt innehåller en förteckning av dessa. Policyn innehåller även hur styrdokumenten ska vara utformade.
\subsection{Syfte}
Syftet med denna policy är att klargöra vilka styrdokument som finns samt funktion och användning av dessa.
\subsection{Omfattning}
Policyn omfattar alla D-sektionens styrdokument.

\subsection{Historik}
Utkast färdigställt av Oliver Levay, Informationsansvarig 2023, till VTM-Extra 2023. Uppdaterad HTM2 2023.
\section{Stadgar}
Senaste versionen av D-sektionens stadgar ska finnas tillgängliga på sektionens hemsida (dsek.se).

\section{Strategiska mål och verksamhetsplan}
Senaste versionen av D-sektionens strategiska mål och verksamhetsplan ska finnas tillgängliga på sektionens hemsida (dsek.se).

\section{Reglemente}
Senaste versionen av D-sektionens reglemente ska finnas tillgängliga på sektionens hemsida (dsek.se).

\section{Policyer}
Policyerna behandlar D-sektionens åsikter i enstaka frågor eller om hur D-sektionen ska agera i olika situationer. Policyer kan därmed ge vägledning till D-sektionen centralt, D-sektionen lokalt i dess
utskott, intresseföreningar och fria föreningar. Vilka som berörs av en policy skall stå i policyn under punkt 1.3 Omfattning. Aktuella policyer ska årligen samt vid ändringar tillställas berörda funktionärer.

\section{Riktlinjer}
Riktlinjer berör administrativa uppgifter samt för hur D-sektionens funktionärer ska agera i olika situationer. Riktlinjer instiftas av styrelsen. Aktuella riktlinjer ska årligen, samt vid ändringar, tillställas berörda funktionärer.

\section{Uppbyggnad av policy}

\subsection{Namn}

En policy ska namnges enligt: \textit{``Policy för X''}.

\subsection{Struktur}

Samtliga av D-sektionens policyer ska inledas på samma sätt som ovan med:

\hfill\begin{minipage}{\dimexpr\textwidth-3cm}
    \xdef\tpd{\the\prevdepth}
    \section*{1. Formalia}
    \subsection*{1.1 Sammanfattning}
    Här ska det finnas en kort sammanfattning av policyn för att läsaren
    snabbt ska förstå vad innehållet i policyn handlar om. \\

    \subsection*{1.2 Syfte}
    Här ska det finnas en beskrivning av vad syftet med policyn är. \\

    \subsection*{1.3 Omfattning}
    Här ska det stå vilka som omfattas av policyn. Exempel är: D-sektionens styrelse, Funktionärer vid D-sektionen.\\

    \subsection*{1.4 Historik}
    Syftet med en historikpunkt är att snabbt kunna få en överblick över hur
    dokumentet ändrats och vem man ska kontakta för att få mer
    information. Här ska det stå vem som gjorde utkastet på det ursprungliga
    förslaget, enligt vilket beslut som det ursprungliga förslaget blev fastställt, vem som gjort eventuella omarbetningar samt enligt vilka beslut de
    eventuella omarbetningarna har blivit fastställda.
    \\

    Historik samt korta beskrivningar av förändringar som skett bör presenteras i en lista i kronologisk ordning. \\
\end{minipage}

\prevdepth\tpd


\section{Uppdatering av styrdokument}
\subsection{Styrdokument som sektionsmötet beslutar om}
Det är endast sektionsmötets som kan
\begin{itemize}
    \item Instifta nya policyer.
    \item Instifta ändringar i befintliga policyer.
    \item Instifta ändringar i reglementet.
    \item Instifta ändringar i sektionens stadgar.
\end{itemize}

\subsection{Styrdokument som styrelsen beslutar om}
Styrelsen kan instifta nya riktlinjer samt uppdatera existerande. Även om en riktlinje instiftas på ett sektionsmöte ägs den av styrelsen. I annat fall bör det vara en policy.

\section{Förteckning}
\subsection{Policyförteckning}
D-sektionen har följande policyer:
\begin{itemize}
    \item Policy för alkohol och droger
    \item Policy för jämlikhet
    \item Policy för mötestider
    \item Policy för ekonomirutiner
    \item Policy för fonder
    \item Policy för hantering av personuppgifter
    \item Policy för röstning
    \item Policy för sektionsbil
    \item Policy för val
    \item Policy för valberedningens arbete
    \item Policy för tackverksamhet
    \item Policy för styrdokument
\end{itemize}
\subsection{Riktlinjeförteckning}
Ändringar i denna förteckning behandlas som redaktionella ändringar. Därmed om styrelsen inför, tar bort eller uppdaterar riktlinjer kan förteckningen ändras utan ett beslut från ett sektionsmöte.

D-sektionen har följande riktlinjer:
\begin{itemize}
    \item Riktlinje för dansplattor
    \item Riktlinje för engelska titlar
    \item Riktlinje för informationsspridning
    \item Riktlinje för hantering av sektionens Facebookssidor
    \item Riktlinje för marknadsföring och prissättning
    \item Riktlinje för sektionsbilens användning
\end{itemize}
\end{document}