\documentclass{dsekkallelse}

\usepackage[T1]{fontenc}
\usepackage[utf8]{inputenc}
\usepackage[swedish]{babel}
\usepackage{multicol}

\setheader{Policy för val}{Policydokument}{}

\title{Policy för val}
\author{Anna Qvil}

\begin{document}

\section{Policy för val}

\subsection{1. Formalia}

\subsubsection{1.1 Sammanfattning}
Denna policyn beskriver hur valprocessen skall gå till både för val gjorda av sektionsmötet och styrelsen. Den täcker även hur processen inför valen skall gå till. 

\subsubsection{1.2 Syfte}
Syftet med denna policy är att förtydliga stadgan och reglementet samt att vägleda valberedningen samt andra tillfälliga valberedningar i sitt arbete

\subsubsection{1.3 Historik}
Utkast färdigställt av: Anna Qvil \\
Ursprungligen antagen enligt beslut: VTM extra 2019.
Uppdaterad: HTM1 2020, S13 2021. 

\subsection{2. Val på sektionsmöten}

\subsubsection{2.1 Utlysning}
Utlysning av de poster som går att söka på sektionsmötet skall anslås senast fyra veckor innan mötet. Valberedningens ordförande ansvarar för att utlysningen sker. 

Utlysningen ska innehålla en beskrivning av posten, mandatperiod, hur man ansöker, sista  ansökningsdag, datum för valet samt vem som kan kontaktas för ytterligare frågor. 

\subsubsection{2.2 Beredning}
Val som uträttas av sektionsmötet bereds av valberedningen med undantag för att kandidater till valberedningen ej valbereds. 

\subsubsection{2.3 Valförfarande}
Motkandidatur mot valberedningens förslag måste anmälas till Talman senast klockan 23.59 två dagar efter valberedningens förslag har publicerats. Ämnar man motkandidera mot fler poster ska alla dessa anmälas.

Motkandidatur skall anslås omgående samt den föreslagna kandidaten skall meddelas. Ansvarig för att meddela kandidaten samt anslå motkandidaten är valberedningens ordförande. Endast personer som blivit valberedda för det berörda valet får motkandidera.
Fri nominering tillåts endast till de poster som valberedningen inte har något förslag till.

På mötet skall val av styrelsemedlem ha 7 minuter för presentation och 8 minuter för frågor. Samt att val av annan förtroendepost skall ha 4 minuter för presentation och 5 minuter för frågor.

\subsection{3. Val på styrelsemöten}

\subsubsection{3.1 Utlysning}
Vid första rekryteringstillfället för en post eller de poster i appendix A samt de av sektionsmötet delegerade valen skall utlysningen av valet ske senast 2 veckor innan mötet då valet skall hållas. 

När på året de olika posterna bör utlysas samt väljas finns specificerat i appendix B. Poster som inte specificeras i appendix B utlyses när utskottsordförande anser det lämpligt. 

Utskottsordförande och i vissa fall utskottsordförande i samråd med utskottsordförande electus ansvarar för att utlysningen sker. Utlysningen skall vara på sådant sätt som ger möjlighet för alla sektionens medlemmar att söka posten. 

Utlysningen ska innehålla en beskrivning av posten, mandatperiod, hur man söker, sista ansökningsdag, datum för valet samt vem som kan kontaktas för mer information. 

För övriga funktionärer kan de väljas in utan särskild utlysning om så utskottsordförande anser det lämpligt. Dock skall processen för dessa även vara på ett sådant sätt att alla har möjlighet att söka posten. 

Vid fyllnadsval av de poster som finns i appendix A skall alltid en utlysning ske. Dock kan styrelsen välja att tillförordna en funktionär till posten under tiden för utlysningen. Övriga poster fyllnadsväljs på så sätt utskottsmästare anser det lämpligt. 

\subsubsection{3.2 Beredning}

Alla de poster som finns i appendix A samt av sektionsmötet delegerade val, med undantag för nollningsfunktionär, skall beredas enligt följande:

Utskottsordförande/utskottsordförande electus sätter ihop en valberedning på minst 3 medlemmar i sektionen bestående av följande: utskottsordförande/utskottsordförande electus själv, minst en gammal utskottsmedlem och minst en medlem i Valberedningen. Denna valberedning skall stadfästas av styrelsen.

Valberedningen för Jubileumsansvarig behöver inte bestå utav en gammal utskottsmedlem.

Utskottsordförande/ utskottsordförande electus agerar valberedningens ordförande i den ovan nämnda valberedningen. 

Den ovan nämnda valberedningen faller under de policys och bestämmelser som finns angående valberedningsarbete. 

Nollningsfunktionärer bereds av Staben. 

Övriga funktionärer bereds på så sätt som utskottsordförande finner lämpligt. 

\subsubsection{3.3 Valförfarande}
Utskottsordförande ansvarar för att skicka in de funktionärer hen önskar välja in som en handling till ett styrelsemöte. På styrelsemötet skall utskottsordföranden redogöra hur processen inför valet gått till. På styrelsemötet kan inte motkandidatur ske utan nyutlysning av valet sker ifall processen anses felaktig eller kandidaterna olämpliga. 

\pagebreak
\subsection{Appendix A: Kärnposter}
\begin{multicols}{2}

\subsubsection{Näringslivsutskottet}
\begin{itemize}
    \item Mentorsansvarig
    \item Medlem i projektgruppen för Teknikfokus 
\end{itemize}

\subsubsection{Källarmästeriet}
\begin{itemize}
    \item Sudo
    \item Root
\end{itemize}

\subsubsection{Aktivitetsutskottet}
 \begin{itemize}
     \item Karnevalsansvarig
     \item Sångarstridsförman
     \item Tandemgeneral
 \end{itemize}
 
\subsubsection{Informationsutskottet}
\begin{itemize}
    \item DWWW-ansvarig
\end{itemize}
 
 \subsubsection{Sexmästeriet}
\begin{itemize}
    \item Hovmästare
    \item Pubmästare
    \item Vice Pubmästare
    \item Barmästare
    \item Vice Barmästare
    \item Sångförman
    \item Köksmästare
    \item Preferensmästare
    \item Vice Köksmästare
\end{itemize}

\subsubsection{Nollningsutskottet}
\begin{itemize}
    \item Stabsmedlem
    \item Øverpeppare
    \item Peppare
    \item Nollningsfunktionär
\end{itemize}

\subsubsection{Framtidsutskottet}
\begin{itemize}
    \item Framtidsledamot
\end{itemize}

\subsubsection{Jubileet}
\begin{itemize}
    \item Jubileumsansvarig
\end{itemize}
\end{multicols}

\pagebreak
\subsection{Appendix B}
\subsubsection{Poster som bör väljas innan början av verksamhetsåret}
\begin{multicols}{2}

\subsubsection{Cafémästeriet}
\begin{itemize}
    \item Stekare
    \item Funktionär
    \item Dagsansvarig
\end{itemize}

\subsubsection{Källarmästeriet}
\begin{itemize}
    \item Root
    \item Sudo
    \item Trädgårdsmästare
\end{itemize}

\subsubsection{Aktivitetsutskottet}
\begin{itemize}
    \item Idrottsförman
    \item LAN-party ansvarig
\end{itemize}

\subsubsection{Informationsutskottet}
\begin{itemize}
    \item DWWW-ansvarig
\end{itemize}

\subsubsection{Sexmästeriet}
\begin{itemize}
    \item Hovmästare
    \item Pubmästare
    \item Vice Pubmästare
    \item Barmästare
    \item Vice Barmästare
    \item Sångförman
    \item Köksmästare
    \item Preferensmästare
    \item Vice Köksmästare
\end{itemize}

\subsubsection{Nollningsutskottet}
\begin{itemize}
    \item Stabsmedlem
    \item Øverpeppare
\end{itemize}

\subsubsection{Framtidsutskottet}
\begin{itemize}
    \item Framtidsledamot 
\end{itemize}
\end{multicols}

\subsubsection{Poster som bör väljas innan läsårets start}

\begin{multicols}{2}
\subsubsection{Näringslivsutskottet}
\begin{itemize}
    \item Medlem i projektgruppen Teknikfokus
\end{itemize}

\subsubsection{Aktivitetsutskottet}
\begin{itemize}
    \item Karnevalsansvarig
\end{itemize}

\subsubsection{Jubileet}
\begin{itemize}
    \item Jubileumsansvarig
\end{itemize}

\end{multicols}





\end{document}