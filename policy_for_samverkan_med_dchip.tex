\documentclass{dsekprotokoll}

\usepackage[T1]{fontenc}
\usepackage[utf8]{inputenc}
\usepackage[swedish]{babel}
\usepackage{multicol}

\setheader{Policy för samverkan med D-Chip}{Policydokument}{}

\title{Policy för samverkan med D-Chip}
\author{Felicia Gabrielii Augustsson}

\begin{document}

\section*{Policy för samverkan med D-Chip}
\section{Formalia}
\subsection{Sammanfattning}
Policy för samverkan med D-Chip beskriver hur sektionen ska samarbeta med sektionens fristående förening ``D-Chip'' (med organisationsnummret 802537-6669).
\subsection{Syfte}
Policyns syfte är att formalisera det stöd som sektionen ämnar att ge D-Chip. 
\subsection{Omfattning}
Sektionen i sin helhet.

\subsection{Historik}

Policyn framtogs av Hampus Serneke till VTM-Extra 2023.

\section{Stöd till D-Chip}

D-Chip får:

\begin{itemize}
    \item boka sektionens lokaler samt sektionsbilen med samma prioritet som sektionens utskott,
    \item utnyttja sektionens alkoholtillstånd eller boka pubar och sittningar efter samverkan med Sexmästaren,
    \item ha en lättillgänglig länk till D-Chips hemsida på sektionens hemsida samt att sektionen hostar D-Chips hemsida,
    \item relevant access på sektionens hemsida för att kunna lägga till evenemang i kalendern,
    \item stöd från sektionen med att hantera mejlaliaser, mejlservrar samt access med LU-kort,
    \item äga dispositionsrätt till Kommitéa,
    \item hjälp med kommunikation med huset och bokning av husets lokaler.
\end{itemize}

Medlemmarna i D-chips styrelse erhåller dessutom de förmåner som ordinarie funktionärer vid D-sektionen får.

\end{document}